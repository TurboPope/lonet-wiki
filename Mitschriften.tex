\documentclass{article}
\usepackage{subscript}
\usepackage{ngerman}
\usepackage{amssymb}
\usepackage{graphicx}
\usepackage{amsmath}
\usepackage[utf8]{inputenc} %Umlaute
\usepackage{listings}
\usepackage{pdfpages}
\usepackage{tkz-graph}
\usepackage{ stmaryrd }
\usepackage{ mathrsfs }
\usepackage{cancel}
\usepackage{tikz}


\GraphInit[vstyle = Shade]
\tikzset{
	LabelStyle/.style = { rectangle, rounded corners, draw,
		minimum width = 2em, fill = yellow!50,
		text = red, font = \bfseries },
	VertexStyle/.append style = { inner sep=5pt,
		font = \Large\bfseries},
	EdgeStyle/.append style = {->, bend left} }


\newcommand{\bild}[4]{ %Pfad, %Weite, %Name, %label
	\begin{figure}[h!]
		\centering
		\includegraphics[width=#2\textwidth]{#1}
		\caption{#3}
		\label{#4}
	\end{figure}	
}

\newcommand{\sieheBild}[4]{
	Siehe Abbildung \ref{#4}
	\bild{#1}{#2}{#3}{#4}
}

\newcommand{\doubleAbs}[1]{
	\|#1\|
}

\newcommand{\floor}[1]{
	\lfloor #1 \rfloor
}
		
\newcommand\circlearound[1]{%
	\tikz[baseline]\node[draw,shape=circle,anchor=base] {#1} ;
}


		
\begin{document}
	
\section{Einführung}
SIEHE FOLIEN!

\section{Vorlesung 1}
\subsection*{Graph}
$G = (V, E), \ E \subseteq V \times V$ \\
$v \in V$ Knoten, $(u,v) \in E$ Kante \\
(abkürzend $uv \in E, \ v \in G, \ uv \in G$)

\subsection*{Ungerichteter Graph}
$uv \in G \Leftrightarrow vu \in G$

\subsection*{Subgraph}
$G = (V_G, E_G)$ ist Subgraph von $H = (V_H, E_H)$, wenn $V_G \subseteq
V_H$ und $E_G \subseteq E_H$ \\
Notation: $G \subseteq H$ \\
$H$ ist Supergraph von $G$

\subsection*{Schnitt von zwei Graphen}
Schnitt von zwei Graphen $G = (V_G,E_G)$ und $H = (V_H,E_H)$ bezeichnet
mit $G \cap H$ besteht aus $V = V_G \cap V_H, \ E = \{ vw: v,w \in V, vw
\in E_G \cap E_H\}$

\subsection*{Vollständiger Graph}
$E = V \times V$

\subsection*{Gewichteter Graph}
$(G,f)$ und $f: E \rightarrow \mathbb{R}$ \\
für $e \in E$ ist $f(e)$ das Gewicht von $e$

\subsection*{Euklidischer Graph}
Knoten sind Punkte im euklidischen Raum \\
Euklidische Distanz zwischen $p$ und $q$: $\| pq \|$ \\
Vollständiger Euklidischer Graph (Verknüpfung der vorherigen Definitionen)

\subsection*{Pfad}
$p = (v_1, v_2, \ldots, v_n)$ von $v_1$ nach $v_n$ in Graphen $G =
(V,E)$ erfüllt $v_i v_{i+1} \in E$ \\
(abkürzend: $v_1 v_2 \ldots v_n, p \in G$)

\subsection*{Verbundener Graph}
$G$ ist verbunden, wenn $\forall v, w \exists$ mind. ein Pfad $p \in G$,
welcher $v$ mit $w$ verbindet

\subsection*{Pfadlänge}
Pfadlänge eines Pfades $p = v_1 \ldots v_n$ in einem gewichteten Graphen
$(G,f)$: \\
$\displaystyle \hat{f}(p) = \sum_{i=1}^{n-1} f(v_i v_{i+1})$ \\

\noindent Euklidische Pfadlänge für Pfad $p$ in Euklidischem Graphen $G$: \\
$\displaystyle \|p\| = \sum_{i=1}^{n-1} \| v_i v_{i+1} \|$ \\

\noindent Topologische Pfadlänge: Graph $G$ gewichtet mit $f(e) = 1$ \\
Für Pfad $p$ in $G$ wird für $f(p)$ auch folgende Notation verwendet: $|p|$

\subsection*{Nachbar}
Für Graph $G$ und $v \in G$ sind die k-hop-Nachbarn von $v$: \\
$N_k(v) = \{ w: \exists p \ \text{der Länge} \leq k \ \text{von} \ v \
\text{nach} \ w \}$ \\
Annahme: $v \in N_k(v)$ (Ausnahmen werden explizit genannt) \\
$N_0(v) = \{v\}, N(v) := N_1(v)$ \\
$v$ und $w$ „sehen sich“, wenn $v \in N(w)$ und $w \in N(v)$

\subsection*{Graph-Klasse $\mathcal{C}$}
„Menge“ von Graphen, die bestimmte Eigenschaft erfüllen

\subsection*{Topologiekontrolle $\tau: \mathcal{C} \rightarrow \mathcal{D}$}
Abbildung von Graph-Klasse $\mathcal{C}$ in Graph-Klasse $\mathcal{D}$ \\
Subgraph-Konstruktion: $\tau(G) \subseteq G$
\\Einschub: 
Topologiekontrolle bezeichnet alle Techniken, die die Menge der aktiven Links zwischen Knoten verringern, um Eigenschaften des Kommunikationsnetzes herzustellen, ohne andere Eigenschaften zu zerstören.

\subsection*{Lokale Sicht}
Die lokale Sicht $G[v,k]$ eines Knotens $v$ aus Graphen $G=(V,E)$
besteht aus: \\
$V_v \in N_k(v)$ \\
$E_v \in \{ xy \in E : x \in N_{k-1}(v) \wedge y \in N_k(v) \}$ \\
$G[v] := G[v,1]$

\section{Vorlesung 2}
\subsection*{k-lokale Topologiekontrolle}
$\tau: \mathcal{C} \rightarrow \mathcal{D}$ ist k-lokale
Topologiekontrolle \\
$\forall G \in \mathcal{C}, \forall v \in \tau (G) : \exists u \in G :
\tau (G)[v] = \tau (G[u,k])[v]$

\subsection*{2D-Geometrie-Definitionen}
\begin{itemize}
	\item Bisektor durch $u$ und $v$ definiert: $B(u,v) = \{x : \|xu\| =
	\|xv\|\}$
	\item Geschlossene Halbebene durch $u$ und $v$ definiert: $H(u,v) = \{x:
	\|xu\| \leq \|xv\| \}$
	\item Kreis mit Radius $r$ um $u$: $C_r(u) = \{ x : \|xu\| \leq r \} $
	\item Den $u,v,w$ umschließenden Kreis
	\item Den kleinsten $u$ und $v$ umschließenden Kreis
	\item Winkel $\measuredangle uvw$ zwischen $uv$ und $vw$
\end{itemize}

\subsection*{Graph-Spanner}
\subsubsection*{Kürzester Weg}
Pfad von $u$ nach $v$ in einem gewichteten Graphen $(G,f)$ ist ein
kürzester Weg wenn für alle Pfade $q$ von $u$ nach $v$ in $G$ gilt:
$f(p) \leq f(q)$ \\
Notation für topologisch kürzesten Weg von $u$ nach $v$ im Graphen $G$:
$\Gamma_G(u,v)$ \\
Notation für Euklidischen kürzesten Weg: $\Pi_G(u,v)$

\subsubsection*{Euklidischer Spanner}
Graph $G$ definiert über Punktmenge $P$ ist ein Euklidischer
$c$-Spanner, wenn für alle $u,v \in G$ gilt: $\displaystyle
\frac{\|\Pi_G(u,v)\|}{\|uv\|} \leq c$

\subsubsection*{Euklidischer Graph-Spanner}
Sei $G$ ein Euklidischer Graph. Ein Subgraph $H \subseteq G$ mit
derselben Knotenmenge ist ein Euklidischer $c$-Spanner von $G$, wenn
$\forall u,v \in G$ gilt: $\displaystyle
\frac{\|\Pi_H(u,v)\|}{\|\Pi_G(u,v)\|} \leq c$

\subsubsection*{Topologischer Graph-Spanner}
Text wie eben. $\displaystyle \frac{|\Gamma_H(u,v)|}{|\Gamma_G(u,v)|}
\leq c$

\subsubsection*{Spanning-Ratio}
$c$ ist die Spanning-Ratio.

\subsection*{Planarer Graph}
Graphentheoretisch: Graph $G$ ist planar, wenn $G$ auf die Ebene
gezeichnet werden kann, sodass sich keine Kanten schneiden. \\
Definition hier: Graph $G$ ist als Zeichnung auf der Ebene gegeben.
Gesucht ist ein Teilgraph $H$, der keine schneidenden Kanten enthält.
Dieser Graph $H$ wird dann als planar bezeichnet.

	
	

\section{Vorlesung 3}
\subsection*{Euklidischer Minimaler Spannbaum}
Punktmenge V, Es sei: $\|G\|$ das Euklidische Gewicht eines Graphen G,
d.h. 
\[
\|G\| := \Sigma_{uv \in G} \|uv\|
\]
Ein Euklidischer minimaler Spannbaum EMST (V) ist ein über V zusammenhängender Graph mit.\\

$\|EMST(V)\| \leq \|G\| \, \forall G $ zusammenhängender Graph über V

\subsubsection*{Beispiel - Konstrutkion nach Kruskal}
Wähle so oft wie möglich unter den noch nicht ausgewählten kürzesten möglichen Kanten eine Kante, die keinen Kreis bildet.

\subsection*{Satz}
Sei V eine Knotenmenge, dann gillt: \\
UDG(V) verbunden $\Rightarrow$ EMST(V) $\subseteq$ UDG(V)

\subsubsection*{Beweis}
Annahme: $\exists e \in EMST(V)$ mit $e \not\in UDG(V)$ \\
$\Rightarrow \|e\| > R$ (R sei UDG-Radius)\\
Betrachte T := EMST(V) $\backslash$ \{e\} \\
EMST(V) ist ein Baum $\Rightarrow$ T ist unverbunden\\
UDG(V) verbunden $\Rightarrow$ $\exists e' \in UDG(V)$ \\
mit $T' := T \cup \{e'\}$ verbunden\\
mit $\|e'\| \leq R < |e|$ \\
$\Rightarrow \|T'\| = \|T\|+\|e'\| < \|T\|+\|e\| = \|EMST(V)\|$\\
$\Rightarrow \lightning$ \\


\subsection*{Satz}
EMST ist keine k-lokale Graphstruktur.\\

\subsubsection*{Beweis}
Annahme: $\exists$ doch k-lokaler Algorithmus für EMST\\
Betrachte folgenden UDG für $a < b$ \\
\sieheBild{Bilder/1.png}{0.7}{bild1}{Bild 1}
\\
Es folgt: Knoten $v_1$ und $v_2$ müssen Kante $v_1 v_2$ entfernen, um EMST zu konstruieren\\
Verändere Position von $v_3$ und $v_4$ sodass $b < a$\\
Die k- Hop - Sicht von $v_1$ und $v_2$ ändert sich nicht.\\
$\Rightarrow v_1 v_2$ wird wie vorhin entfernt \\
$\Rightarrow$ Somit kann der Algorithmus keinen EMST konstruieren.

\subsection*{Lokaler Minimaler Spannbaum (LMST)} 
Idee: wende EMST-Konstruktion nur auf die Nachbarn eines Knotens an, also: 
\[
	uv \in LMST(V) \Leftrightarrow uv \in EMST(N(u))
\]
Beachte: Dieser Graph ist i.a. gerichtet; Beispiel: 
\\
\sieheBild{Bilder/2.png}{0.4}{bild2}{Bild 2}
\\
EMST(N(u)) = \{uv, $uw_4$\}\\
EMST(N(v)) = \{$vw_1$, $w_1 w_2$, $w_3 w_4$, $w_4 u$\}
\\
Somit existiert in LMST(V) beispielsweise uv aber vu existiert nicht.
\begin{itemize}
	\item Um diesen Graphen noch ungerichtet zu konstruieren, definiert man: 
	\[
	uv \in LMST(V) \Leftrightarrow uv \in EMST(N(u))
	\]
	\[
	und
	\] 
	\[
	vu \in EMST(N(v))
	\]
\end{itemize}

\subsection*{Gleiche Kantenlängern verhindern LMST}
Betrachte: \\
\sieheBild{Bilder/3.png}{0.4}{bild3}{Bild 3}\\
$\|uv\| = \|vu\| = \|wu\|$
\\
EMST(\{u, v, w\}) ist nicht eindeutig
\\
Beispielsweise: 
\begin{itemize}
	\item EMST(N(u)) = \{uw, wv\}
	\item EMST(N(v)) = \{vu, uw\}
	\item EMST(N(w)) = \{wv, wu\}
\end{itemize}

\noindent Nach vorheriger Definition enthielte LMST damit gar keine Kante.\\
Wir müssen im Falle gleich langer Kanten eine der beiden Kanten eindeutig "`bevorzugen"'. Definiere hierzu folgende Relation $>$ : \\
$
uv > xy \Leftrightarrow \|uv\| > \|xy\|
$\\
\hspace*{10mm}oder 
$\|uv\| = \|xy\|$ und $max\{u, v\} > max\{x, y\}$\\
\hspace*{10mm}oder 
$\|uw\| = \|xy\|$ und $max\{u, v\} = max\{x, y\}$ und $min\{u, v\} > min\{x, y\}$
\\
(Hierbei sei Für min und max Berechnung irgendeine lexikographische Ordnung auf der Knotenmenge definiert)

\subsubsection*{Übung}
$\neg (uv > xy)$ und $\neg (xy > uv) \Leftrightarrow uv = xy$

\subsection*{Satz: Vorbereitung}
Im folgenden betrachten wir EMST Konstruktion unter dieser Kantenordnung $>$ und die vorige LMST Definition\\ 
$uv \in LMST(V) \Leftrightarrow uv \in EMST(N(u)) $
und 
$vu \in EMST(N(v))$

\subsection*{Satz}
LMST ist eine 2 lokale Topologiebildung (von UDG-Graphen nach LMST Graphen)

\subsubsection*{Beweis}
Zu zeigen ist: $\forall G \in UDG, \forall v \in LMST (G): \exists u \in G:$\\
$LMST(G)[v] = LMST(G[u, 2])[v]$\\
\vspace*{5mm}
Es ist $uw \in LMST(G)[v]$ \\
$\Leftrightarrow vw \in EMST(N_G(v))$ und $wv \in EMST(N_G(w))$ (A)\\
$[$Notation: $N_G(v) = $ Nachbarn von v in Graphen G $]$\\
Mit $N_G(v) = N_{G[v,2]}(v)$ und  $N_G(2) = N_{G[v,2]}(w)$ folgt\\
(A) $\Leftrightarrow vw \in EMST(N_{G[v,2]}(v))$ und $vw \in EMST(N_{G[v,2]}(w))$\\
$\Leftrightarrow vw \in LMST(G[v,2])$\\
$\Leftrightarrow vw \in LMST(G[v,2])[v]$ \\
(Somit existiert mit u = v ein solches u)

\section{Vorlesung 4}

\subsection*{Satz}
Für zusammenhängenden UDG gilt:
$EMST \subseteq  LMST$

\subsubsection*{Beweis}
Annahme $uv \not\in LMST$ (Zeige $uv \not\in EMST$)\\
$\Rightarrow uv \not\in EMST(N(u))$ oder $vu \not\in EMST(N(v))$\\
Betrachte oBdA $uv \not\in EMST(N(u))$\\
Betrachte zwei Fälle:
\begin{enumerate}
	\item $uv \not\in UDG \Rightarrow^{EMST\subseteq UDG} uv \in EMST$
	\item $uv \in UDG$,\\
	da v ein Nachbar von u ist, gilt $\exists p \in EMST(N(u))$ mit $p = p_1 p_2 ... p_n$ und $p_1 = u, p_n = v$
\end{enumerate}
\sieheBild{Bilder/4.png}{0.4}{bild4}{Bild 4}\\
Es gilt: $\|uv\| > \|p_i p_{i+1}\|$\\
$[$ Annahme $\|uv\| < \|p_i p_{i+1}\|]$:\\
Streiche $p_i p_{i+1}$ in EMST(N(u)) und füge Kante uv ein.\\
Ergebnis ist zusammenhängender Graph T in N(u) mit $\|T\| < \|EMST(N(u))\|$ \\
$\lightning$\\
\vspace*{5mm}\\
Betrachte $p_i p_{i+1}$:\\
Entweder ist $p_i p_{i+1}$ auch Kante in EMST(V) oder falls nicht dann gilt: \\
$\exists$ Pfad $q = q_1 ... q_n$ mit $q_1 = p_i$ und $q_n = p_{i+1}$ und $q_j q_{j+1} \in EMST(V)$ und $\|q_j q_{j+1}\| < \|p_i p_{i+1}\|$\\
Beweis analog zu  (*)
\sieheBild{Bilder/5.png}{0.4}{bild5}{Bild 5}\\
$\Rightarrow$ u und v sind in EMST(V) über einen Pfad $p'$ verbunden mit $\forall e \in p': \|e\| < \|uv\|$\\
$\Rightarrow$ da EMST(V) ein Baum ist (d.h. keine Zyklen emthält) folgt: $uv \not\in EMST(V)$

\subsection*{Satz}
LMST hat einen maximalen Grad von 6.

\subsubsection*{Beweis}
Übung

\subsection*{Relativer Nachbarschaftsgraph (RNG)}
Definition: Sei V eine Punktemenge. RNG(V) ist definiert als:
\[
	uv \in RNG(V) \Leftrightarrow \doubleAbs{uv} \leq max\{\doubleAbs{uw}, \doubleAbs{vw}\} \forall w \in V \backslash \{u,v\}
\]

\subsubsection*{Alternative Definition}
\sieheBild{Bilder/6.png}{0.7}{bild6}{Bild 6}\\

\subsection*{Definition: Unit - Relativer - Nachbarschaftsgraph: URNG}
URNG(G) = RNG(V) $ \cap $ UDG(V)

\subsubsection*{Übug}
URNG ist eine 1- lokale Graphkonstruktion\\
$[ URGNG \subseteq UDG, URNG \subseteq RNG]$

\subsection*{Satz}
$LMST(V) \subseteq URNG(V)$

\subsubsection*{Beweis}
Annahme: $\exists uv \in LMST(V) mit uv \not\in URNG(V)$
\\
\sieheBild{Bilder/7.png}{0.4}{bild7}{Bild 7}\\
Es folgt: $\exists w : \doubleAbs{uw}, \doubleAbs{vw} < \doubleAbs{uv}$\\
Betrachte EMST(N(u)):\\
$uv \in LMST(V) \Rightarrow uv \in EMST(N(u))$\\
EMST(N(u)) \textbackslash \{uv\} ist unverbunden\\
und entweder uw oder vw verbindet diesen Graphen wieder.\\
oBdA sei dies die Kante uw.\\
Konstruiere folgenden zusammenhängenden Graphen
\\
T := EMST(N(u)) \textbackslash \{uv\} $\cup$ \{uw\}\\
Es folgt: $\doubleAbs{T} = \doubleAbs{EMST(N(u))} - \doubleAbs{uv} + \doubleAbs{uw} < \doubleAbs{EMST(N(u))}$\\
$ \Rightarrow \lightning$

\subsection*{Satz}
Sei V eine Menge mit n Punkten. Es gilt:\\
$\frac{\doubleAbs{\Pi_{URNG}(u,v)}}{\doubleAbs{\Pi_{UDG}(u,v)}} \leq n-1 $

\subsubsection*{Beweis}
\begin{enumerate}
	\item $\doubleAbs{uv} > R (UDG-Radius)$ \\
	Somit ist $\doubleAbs{\Pi_{URNG}(u,v)} \geq \doubleAbs{uv} > R (1)$\\
	Es gilt immer: $|\Pi_{URNG}(u,v)| \leq n-1$\\
	Jede besuchte Kante hat maximale Länge R (R=UDG-Radius)\\
	Somit ist $\doubleAbs{\Pi_{UDG}(u,v)} \leq (n-1) * R (2)$\\
	Somit ist mit (1) und (2):
	\\
	$\frac{\Pi_{URNG}(u,v)}{\doubleAbs{\Pi_{UDG}(u,v)}} \leq \frac{\not R * (n-1)}{\not R} = n-1$
	\item Betrachte den Pfad p von u nach v in EMST(V)
	\\
	\sieheBild{Bilder/8.png}{0.4}{bild8}{Bild 8}\\
	\\
	Jede Kante in P hat eine Länge $\leq \doubleAbs{uv}$\\
	(ansonsten wäre EMST nicht minimal; tausche diese Kante $> \doubleAbs{uv}$ mit $\doubleAbs{uv}$ aus)\\
	Somit ist $\doubleAbs{p} \leq \doubleAbs{uv} (n-1)$
	\\
	Mit EMST(V) $ \subseteq$ LMST(V) $\subseteq$ URNG(V) folgt:\\
	$p \in URNG(V)$, also\\
	$\doubleAbs{\Pi_{URNG}(u,v)} \leq \doubleAbs{P} \leq \doubleAbs{uv}(n-1)$\\
	Mit $\doubleAbs{\Pi_{UDG}(u,v)} = \doubleAbs{uv}$ folgt dann\\
	$\frac{\doubleAbs{\Pi_{URNG}(u,v)}}{\Pi_{UDG}(u,v)} \leq \frac{\not\doubleAbs{uv}* (n-1)}{\not\doubleAbs{uv}} = n-1$
\end{enumerate}

\section{Vorlesung 5}
\subsection*{Ergänzungen}
Es gibt Knotenpositionierungen für die die euklidische SPanning - Ratio n-1 erreicht wird.
\sieheBild{Bilder/10.png}{0.4}{bild10}{Bild 10}\\
Ergänzung zu Bild: v1 und vn schauen unten als Kanten heraus.
Die einzigen in RNG verblöeibenden Kanten sind $v_i, v_{i+1} \forall 1 \leq i \leq n-1$\\
Sei: $\doubleAbs{v_1 v_n} = R$ (R ist UDG-Radius)\\
Damit ist die Paflänge von $v_1 v_2 ... v_n$ gegeben wiefolgt
\[
	\doubleAbs{v_1 v_2 ... v_n} \approx (n-1) * R
\]
Außerdem ist $v_1 v_2 ... v_n$ ein gerichteter Pfad zwischen $v_1$ und $v_n$ im RNG.\\
Somit folgt
\[
	\frac{\doubleAbs{\Pi_{RNG}(v_1 v_n)}}{\doubleAbs{\Pi_{UDG}(v_1 v_n)}} = \frac{\doubleAbs{v_1 v_2 v_n}}{\doubleAbs{v_1 v_n}} \approx \frac{(n-1) * R}{R}
\]
Wir sagen hierzu: Die Schranke n-1 ist scharf ud schreiben URNG ist ein $\Theta(n)$-Unit-Disk-Graph-Spanner (n ist die Anzahl Knoten)\\
Der RNG ist nicht gradbeschränkt\\
\sieheBild{Bilder/11.png}{0.4}{bild11}{Bild 11}\\
Jede Kante $v_1 v_i$ für $2 \leq i \leq n$ ist in RNG enthalten.

\subsection*{Gabriel Graph}
Definition: Der Gabriel Graph GG(V) für eine Punktmenge V besteht aus genau den Kanten uv für die folgendes gilt:
\\
C(u,v) enthält keine Punkte aus $V\backslash\{u,v\}$\\
\sieheBild{Bilder/12.png}{0.4}{bild12}{Bild 12}\\
\\
Auch hier wieder eine lokale Variante.\\

\subsubsection*{Definition}
UGG(V) = GG(V) $\cap$ UDG(V)\\
Es gilt: UGG ist 1-lokale Topologiekontrolle\\
$[UGG \subseteq UDG, UGG \subseteq GG]$\\
Es gilt: $URNG \subseteq UGG$\\
$[RNG \subseteq GG]$
\\
Es gibt Punktmengen V für die UGG(V) eine euklidische Spannung-Ratio von $\frac{(\lfloor \frac{n}{2}\rfloor-1)}{\sqrt{n}}$ erreicht.\\
\sieheBild{Bilder/13.png}{0.4}{bild13}{Bild 13}\\
Hier kann man zeigen, dass nur die Kreise $C(p_i, p_{i+1})$ leer sind.\\
Damit ist der einzig mögliche Pfad von $p_1$ nach $p_n$ in GG durch $p_1 p_2 p_3$ $p_{1/2} p_{1/2+1}$ ... $p_n$ gegeben.\\
Man rechnet leicht nach:
\[
	\doubleAbs{p} = 2 * \frac{\lfloor n/2 \rfloor -1}{\sqrt{n}}
\]
Somit gilt für $p_1$ und $p_n$ in UDG mit UDG-Radius R = 2:
\[
	\frac{\doubleAbs{\Pi_{UGG}(p_1, p_n)}}{\doubleAbs{\Pi_{UDG}(p_1, p_n)}} = \frac{2 * \frac{\lfloor n/2 \rfloor -1}{\sqrt{n}}}{2} = \frac{\lfloor n/2 \rfloor -1}{\sqrt{n}}
\]
Wir sagen auch UGG ist ein Euklidischer $\Omega(\frac{\floor{n/2}-1}{\sqrt{n}}) = \Omega(\sqrt{n})$
Unit-Disk-Graph Spanner\\
Frage: Ist GG auch ein euklidischer $O(\sqrt{n})$ -UDG-Spanner (d.h. $\Theta(\sqrt{n})$-UDG-Spanner)

\subsection*{Delanay-Triangulation}
Gegeben sei Punktmenge S.\\
Voronoi-Region um $u \in S$:
\[
	VR_S(u) = \bigcap\limits_{v \in S \backslash \{u\}}{} H(u,v)
\]
Beispiel:\\
\sieheBild{Bilder/14.png}{0.4}{bild14}{Bild 14}\\

\subsubsection*{Voronoi-Diagramm}
zu einer Punktemenge S = \{$u_1, ..., u_n$\}
\[
	VD(S) = \{VR_S(u_1), VR_S(u_2), ..., VR_S(u_n)\}
\]
Beispiel:
\sieheBild{Bilder/15.png}{0.4}{bild15}{Bild 15}\\
\sieheBild{Bilder/16.png}{0.4}{bild16}{Bild 16}\\
Voronoi-Diagramm ist nicht degeneriert, wenn alle Voronoi-Punkte nichtdegeneriert sind.

\subsubsection*{Definition: Delonoy-Triangulation zu Punktmenge S (Del(S))}
Definition 1: Betrachte VD(S). Verbinde alle Punkte in S, deren Voronoi-Region eine gemeinsame Kante haben.
\sieheBild{Bilder/17.png}{0.4}{bild17}{Bild 17}\\
$[uv \in Del(S) \Leftrightarrow VR_S(u) \cap VR_S(v) \neq \phi]$
\\
Definition 2: Verbinde $u,v,w \in S$ mit einem Dreieck, wenn C(u,v,w) $\cap$ S = \{u,v,w\}
 (d.h. enthält keine weiteren Punkte)
 \\
\sieheBild{Bilder/18.png}{0.4}{bild18}{Bild 18}
 
 Definition 3: Verbinde $u,v \in S$ mit Kante, wenn ein Kreis C existiert u und v liegen auf Kreisbogen von C und $C \cap S = \{u,v\}$
 \\
\sieheBild{Bilder/19.png}{0.4}{bild19}{Bild 19}
 

\section{Vorlesung 6}
(Papier) \\
Kanten nach Def. 1: $v_1v_2, v_2v_3, v_3v_4, v_4v_1$ \\
Kanten nach Def. 2: $\emptyset$ \\
Kanten nach Def. 3: wie Def. 1 \\
Vereinfachende Annahme 1: Nicht-Cozirkularität; d.h. Punkte sind so
angeordnet, dass keine vier Punkte auf einem Kries liegen \\
Übung: Nihct-Cozirkularität $\Leftrightarrow$ Voronoi-Diagramm ist
nicht-degeneriert \\
Beobachtung 2: (Papier) \\
Kanten nach Def. 1: $v_1v_2, v_2v_3, v_3v_4$ \\
Kanten nach Def. 2: $\emptyset$ \\
Kanten nach Def. 3: wie Def. 1 \\
Vereinfachende Annahme 2: Nicht-Colinearität; d.h. keine drei Punkte
liegen auf einer Geraden \\
Übung: Für Punktmenge $S$ mit $|S| \geq 3$: nicht-Colinearität
$\Leftrightarrow$ keine der Voronoi-Knoten ist eine unendliche Gerade \\
Übung: Unter der Annahme der Nicht-Cozirkularität und Nicht-Colinearität
sind die Delaunnay-Definitionen 1, 2 und 3 äquivalent \\
\subsection*{Satz} Delaunny-Triangulierung ist planar

\subsection*{Diskussion}
Bisher $URNG := RNG \cap UDG$ \\
$UGG := GG \cap UDG$ (sind lokale Topologien) \\
Jetzt: $UDel := Del \cap UDG$ ist das auch eine lokale Topologie?
$\rightarrow$ nein! Übung!
\begin{enumerate}
	\item $Del$ konstanter Spanner
	\item daraus folgt nicht automatisch, dass $UDel \subseteq Del$ ein
	konstanter UDG-Spanner ist; aber: Idee Spannerbeweis zu Del auf UDel
	anpassen, um damit zu zeigen, dass UDel ein konstanter UDG-Spanner ist
	\item werden Graphkonstrukte $G$ auf Basis von Del anschauen mit der
	Eigenschaft: \begin{enumerate}
		\item $G$ ist lokale Topologie
		\item $UDel \subseteq$ G
	\end{enumerate}
\end{enumerate}

\subsection*{Satz}
$Del(S)$ ist ein $\frac{2\pi}{3\cos(\frac{\pi}{6})}$-Spanner ($=
\frac{4\sqrt{3}}{9}\pi \approx 2,42$) \\
Beweisidee: \\

\subsection*{Lemma}
betrachte Punktmenge $S$ (nicht cozirkulär und nicht colinear).
Zeige zunächst für alle Knotentripel $(u,v,w) \in S^3$ mit folgender
Eigenschaft: (Papier) \\
Länge des kürzesten Pfades $DT(u,v)$ in $Del(S)$ von $u$ nach $v$ ist
beschränkt durch die Länge des oberen Kreisbogens von $u$ nach $v$. (die
gestrichelte Linie) \\
Beweisführung: Sortiere alle so konstruierten Kreise bzgl. des Winkels
$\Theta$ aufsteigend und zeige die Aussage per Induktion: \\
I.A.: für Kreis mit kleinstem Winkel $\Theta$ zeigt man, dass $uv \in
Del(S)$ gilt. \\
I.S.: (Papier) betrachte den Punkt $t$ im oberen Kreisabschnitt, welcher
den Winkel $\angle utv$ maximiert. (wenn kein solches t existiert, dann
ist $uv \in Del(S)$) \\
Betrachte die zwei kleineren Kreise $C_1$ und $C_2$, für die wir zeigen,
dass die Induktionsvoraussetzung schon anwendbar ist. \\
Aus den kürzesten Weglängen für $C_1$ und $C_2$ leiten wir die kürzeste
Weglänge für $C$ her.

\section{Vorlesung 7}
\subsection*{Rückblick}
\sieheBild{Bilder/20.png}{0.4}{bild20}{Bild 20}
Del(S)\\
$UDel(S) \subseteq G = Del(S) \cap UDG(S)$\\
$UDel(S) \subseteq Del(S)$
Del(S) 2,42-Spanner\\
\sieheBild{Bilder/21.png}{0.4}{bild21}{Bild 21}
$\doubleAbs{DT(u,v)} \leq r \cdot \Theta$\\
\sieheBild{Bilder/22.png}{0.4}{bild22}{Bild 22}

\subsection*{Beweis Fortsetzung}
\sieheBild{Bilder/23.png}{0.4}{bild23}{Bild 23}
$\doubleAbs{pH} > \doubleAbs{pq} = R$\\
$\lightning$\\
$\doubleAbs{DT(p,q)} \leq \doubleAbs{DT(p, t)}+ \doubleAbs{DT(t, q)}$\\
Mögliche Reperatur des Beweises:\\
\begin{enumerate}
	\item Fall: $\doubleAbs{pt} \leq \doubleAbs{pq}$
	\item Fall: $\doubleAbs{pt} > \doubleAbs{pq}$\\
	Hier ist nicht notwenigerweise $pt \in UDG(S)$\\
\end{enumerate}
\sieheBild{Bilder/24.png}{0.4}{bild24}{Bild 24}
Für Fall 2 verschiebe Kreis so, dass oberer Bereich leer wird;
damit wird der Teil des Kreises im unteren Bereich größer. 
Wenn unterer Berech immer noch leer, dann fertig. Ansonsten zeige, dass nur noch der Fall 1 eintreten kann.

\subsection*{Satz}
Sei S eine Punktmenge in der Ebene.
Für beliebige Punkte p und q aus S gilt:
$\frac{\doubleAbs{DT(p, q)}}{\doubleAbs{pq}} \leq \frac{2 \pi}{3 cos(\pi / 6)} = \frac{4 \sqrt{3}}{9} \pi \approx 2,42$\\
Beweis: Es sei L die Gerade durch p und q.
\sieheBild{Bilder/25.png}{0.4}{bild25}{Bild 25}
$C_1$ und $C_2$ seien Kreise durch p und q mit Kreiszentren $O_1, O_2$ oberhalb bzw. unterhalb von L udn den Winkeln 
$\measuredangle qp O_1 = \measuredangle qp O_2 = \measuredangle pq O_1 = \measuredangle pq O_2 = \frac{\pi}{6}$\\
\sieheBild{Bilder/26.png}{0.4}{bild26}{Bild 26}
Und Radius von $C_1, C_2$ sind $\doubleAbs{p O_1} = \doubleAbs{q O_1} = \doubleAbs{p O_2} = \doubleAbs{q O_2} = r_1$\\
\sieheBild{Bilder/27.png}{0.4}{bild27}{Bild 27}
\begin{enumerate}
	\item Fall: LUNE ist leer\\
	Dann ist $C_1$ unterhalb von L leer. Mit vorherigem Lemma folgt: \\
	$\doubleAbs{DT(p, q)} \leq r_1 \cdot \Theta$\\
	Es ist $\Theta_1 = \frac{4}{3} \pi$ und $r_1 = \frac{\doubleAbs{pq}}{2 cos(\pi / 6)}$\\
	Somit ist. $\doubleAbs{DT(p,q)} \leq \frac{4}{3} \pi \cdot \frac{\doubleAbs{pq}}{R cos(\pi / 6)} = \frac{2 \pi}{3 cos(\pi / 6)} \cdot \doubleAbs{pq}$
	\sieheBild{Bilder/28.png}{0.4}{bild28}{Bild 28}
	\item Fall: Lune ist nicht leer \\
	oBda sei die Menge $S' \subseteq S$ der Punkt x in der Lune mit $\doubleAbs{px} \leq \doubleAbs{qx}$ nicht leer\\
	Es sei $\mathscr{C}$ die Menge der Kreise mit p liegt auf dem Kreisbogen und Kreismitteöpunkt liegt auf der Strecke $O_1 P$\\
	\sieheBild{Bilder/29.png}{0.4}{bild29}{Bild 29}
	oBdA existiere ein Punkt $t \in S'$, der nicht unterhalb von L liegt, sodass ein Kreis $G_3 \in \mathscr{C}$ existiert mit: 
	\begin{itemize}
			\item t liegt auf dem Kreisbogen von $G_3$
			\item $C_3$ enthält keine weiteren Punkte aus S
	\end{itemize}
	\sieheBild{Bilder/30.png}{0.4}{bild30}{Bild 30}
	$O_3$ sei der Mittelpunkt von $C_3$\\
	Es sei : $\Theta = \measuredangle qpt, \varphi = \measuredangle pqt$\\
	c = Distanz von t zum nächsten Punkt $t'$ auf L\\
	$\alpha$ = oberer Winkel von $\measuredangle p O_3 t$\\
	r = Radius von $G_3$\\
	Beweis läuft nun per Induktion über alle möglichen Knotenpare (p,q) aufsteigend sortiert bzgl. $\doubleAbs{pq}$\\
	I.A.: Für $\doubleAbs{pq} \leq \doubleAbs{xy} \forall x, y \in S$ ist offensichtlich C(p,q) leer.\\
	\sieheBild{Bilder/31.png}{0.4}{bild31}{Bild 31}
	Somit gibt es einen leeren Kreis um p und q und somit ist $pq \in Del(S)$\\
	Also $\frac{\doubleAbs{DT(p, q)}}{\doubleAbs{pq}} = \frac{\doubleAbs{pq}}{\doubleAbs{pq}} = 1\leq \frac{2  \pi }{3 cos(\pi / 6)}$\\
	I.S.: Betrachte pq. Nach Induktionsannahme gilt die Behauptung für alle kürzeren Kanten.\\
	Es gilt stets: \\
	$\doubleAbs{DT(p, q)} \leq \doubleAbs{DT(p, t)} + \doubleAbs{DT(t, q)}$\\
	\sieheBild{Bilder/32.png}{0.4}{bild32}{Bild 32}
	Da $C_3$ unterhalb von pt leer ist, folgt mit vorherigem Lemma $\doubleAbs{DT(p, t)} \leq \alpha \cdot r$\\
	Nach Induktionsannahme gilt: $\doubleAbs{t, q} \leq \frac{2 \pi}{3 cos(\pi / 6)} \cdot \doubleAbs{tq}$
	
	
\end{enumerate}



\section{Vorlesung 8}
Falls $t$ auf $L$ liegt: $\alpha = \frac{4}{3}\pi$ und $\displaystyle r
= \frac{\|pt\|}{2\cos(\frac{\pi}{6})}$ \\
$\displaystyle \|DT(p,q)\| \leq
\frac{2\pi}{3\cos(\frac{\pi}{6})}(\|pt\|+\|tq\|) =
\frac{2\pi}{3\cos\frac{\pi}{6}}\|pq\|$ \\
Falls $t$ oberhalb von $L$ liegt: $\alpha = \ldots = \frac{4}{3}\pi -2
\Theta$ \\ und mit Sinus-Satz $\displaystyle \|pt\| =
\frac{c}{\sin\Theta}$ \\
\ldots \\
$\displaystyle r = \frac{c}{2\sin(\frac{\pi}{3}+2\Theta)\cdot\sin\Theta}$\\
Für $\|tq\|$ gilt ebenfalls: $\displaystyle \|tq\| =
\frac{c}{\sin\varphi}$ \\
Substitution von $\|tq\|, r, \alpha$ in (*) ergibt: $\displaystyle
\|DT(p,q)\| \leq
2(\frac{2}{3}\pi-\Theta)\cdot\frac{c}{2\sin(\frac{\pi}{3}+2\Theta)\cdot\sin\Theta}+\frac{2\pi
	c}{3\cos(\frac{\pi}{6}\cdot\sin\varphi)}$ \\
Es sei $a$ die Distanz von $p$ nach $t'$ und $b$ die Distanz von $t'$
nach $q$. Dann gilt: $\displaystyle \|pq\| = a+b = \frac{c}{\tan\Theta}
+ \frac{c}{\tan\varphi}$ \\
Somit ist insgesamt: $\displaystyle \frac{\|DT(p,q)\|}{\|pq\|} \leq
\underbrace{\frac{(\frac{2}{3}\pi-\Theta)/(\sin(\frac{\pi}{3}+\Theta)\cdot\sin\Theta)+2\pi/(3\cos(\frac{\pi}{6})\cdot\sin\varphi)}{\frac{1}{\tan}\Theta+\frac{1}{\tan}\varphi}}_{f(\Theta,\varphi):=}$
\\
und da $t$ innerhalb der LUNE, strikt oberhalb von $L$ und $t$ näher zu
$p$ als $q$, folgt: $\varphi > 0, \Theta \geq \varphi$ und
$\displaystyle \Theta+\varphi < \frac{\pi}{3}$ \\
Mit Kurvendiskussion erhält man für die Funktion $f(\Theta,\varphi)$ und
die Constraints $\varphi > 0, \Theta \geq \varphi$ und $\displaystyle
\Theta+\varphi < \frac{\pi}{3}$: $\displaystyle
\underset{0<\varphi\leq\Theta<\frac{\pi}{3}-\varphi}{f(\Theta,\varphi)}
\leq \frac{2\pi}{3\cos(\frac{\pi}{6})}$ \\
Somit folgt insgesamt $\displaystyle \frac{\|DT(p,q)\|}{\|pq\|} \leq
\underset{0<\varphi\leq\Theta<\frac{\pi}{3}-\varphi}{f(\Theta,\varphi)}
\leq \frac{2\pi}{3\cos(\frac{\pi}{6})}\Box$ \\
\subsection*{Direkter DT Pfad von u nach v}
Definition: direkter DT-Pfad von $u$ nach $v$\\
Sei $S$ eine Punktmenge und $u,v \in S$: (Papier)
\begin{itemize}
	\item Betrachte $VD(S)$
	\item Strecke von $u$ nach $v$ schneidet $Vor(b_0=u), Vor(b_1), \ldots
	Vor(b_{k-1}=v)$
	\item $Vor(b_i)$ und $Vor(b_{i+1})$ sind immer benachbart
	\item d.h. $b_ib_{i+1}$ bilden Kante in $DT(S)$ bzw. $b_0b_1 \ldots
	b_{k-1}$ ist ein Pfad in $DT(S)$
	\item dieser Pfad ist der \underline{direkte} DT-Pfad von $u$ nach $v$.
	\item direkter DT-Pfad ist \underline{einseitig} wenn alle Pfadknoten
	oberhalb (bzw. unterhalb) von Geraden durch $u$ und $v$ liegen
\end{itemize}
Beobachtung (Beweis Übung): Einseitiger DR-Pfad von $u$ nach $v$ ist
bzgl. der euklidischen Pfadlänge nach oben durch $\displaystyle
\frac{\pi}{2}\|uv\|$ beschränkt (Papier) \\
\subsection*{Lemma}
für alle direkten DT-Pfade $u=b_0,b_1,\ldots,b_{k-1}=v$ gilt:
alle $b_i \in C(u,v)$\\ Beweis (siehe Dobkin et. al.) (Papier) \\
Korollar: $uv \in UDG(S) \Leftarrow$ direkter DT-Pfad von $u$ nach $v$
besucht nur Kanten aus
 $UDG(S)$\\
  Beweis: offensichtlich (Papier) jede
besuchte Kante $b_{i-1}b_i$ erfüllt $\|b_{i-1}b_i\| \leq \|uv\| \leq$
UDG-Radius \\
Dobkin-Beweis der Spanning-Ratio $\displaystyle \frac{1+\sqrt{5}}{2}\pi$
der Delannay-Triangulierung: Konstruiert wird folgender Pfad $\Pi(u,v)$
mit der Eigenschaft $\displaystyle \frac{\|\Pi(u,v)\|}{\|uv\|} \leq
\frac{1+\sqrt{5}}{2}\pi$ \\
Betrachte direkten DT-Pfad von $u$ nach $v$ (Papier)
\begin{itemize}
	\item Solange $DT(u,v)$ oberhalb von $L$ ist, verwende diesen Pfadabschnitt
	\item Für jede Kante $b_ib_{i+1}$ des DT-Pfades, welche $L$ schneidet,
	suche den nächsten Knoten $b_j$ welcher wieder oberhalb von $L$ liegt
	und verbinde $b_i$ und $b_j$ mit folgender Abkürzung: \begin{enumerate}
		\item \underline{entweder} wird der direkte DT-Pfad von $b_i$ nach $b_j$
		gewählt \underline{oder}
		\item es wird ein Pfad über folgende Knotenmenge konstruiert $M :=
		\{q\in S : x(b_i) \leq x(q) \leq x(b_j)$ und q liegt oberhalb von L und
		unterhalb von $b_ib_j\}$
	\end{enumerate}
\end{itemize}

\section{Vorlesung 9}
\subsection*{Wiederholung}
\sieheBild{Bilder/33.png}{1.0}{bild33}{Bild 33}
Für (1) gilt: $\doubleAbs{b_i b_j} \leq \doubleAbs{uv}$\\
$b_i, b_j \in C(u,v)$\\
Somit folgt für  $uv \in UDG(S)$, dass auch 
$b_i, b_j \in UDG(S)$ \\
Für $(2) gilt:$ Die Menge aller Punkte in T ist auch in C(u,v). Damit ist für jedes $(x,y) \in T^2: \doubleAbs{xy} \leq \doubleAbs{uv}$\\
analog: $uv \in UDG \Rightarrow xy \in UDG$\\
Insgesamt ist der im Dobkin-Beweis konstruierte Pfad $\Pi(u,v)$ auch ein Pfad im UDG, wenn $uv \in UDG$

\subsection*{Satz}
Für $u,v \in S$ (S ist Punktmenge) gilt:\\
\[
	\frac{\doubleAbs{\Pi_{UDel(S)}(u,v)}}{\doubleAbs{\Pi_{UDG(S)}(u,v)}} \leq \frac{1+\sqrt{5}}{2}*\pi
\]

\subsubsection*{Beweis}
Sei $\Pi_{UDG}(u,v) = v_0 v_1 ... v_n, v_0 = u, v_n = v$ \\
der Euklidisch kürzeste Pfad von u nach v in UDG.
\\
Für jede Kante $v_i v_{i+1}$ goibt es mit weniger Überlegung einen Pfad $\Pi(v_i, v_{i+1})$ in UDel(S) mit \[
	\doubleAbs{\Pi(v_i, v_{i+1})} \leq \frac{1+ \sqrt{5}}{2} * \pi * \doubleAbs{v_i v_{i+1}}
\]
Somit ist \[
	\doubleAbs{\Pi_{UDel(S)}(u,v)} \leq \doubleAbs{\Pi(v_0, v_1) \Pi(v_1, v_2) ... \Pi(v_{n-1}v_n))} = \sum\limits_{i = 0}^{n-1} \doubleAbs{\Pi(v_i, v_{i+1})} \leq  \sum\limits_{i = 0}^{n-1} \frac{1 + \sqrt{5}}{2} * \pi * \doubleAbs{v_i v_{i+1}}
	= \frac{1+\sqrt{5}}{2} * \pi *  \sum\limits_{i = 0}^{n-1} \doubleAbs{v_i v_{i+1}}
	= \frac{1 + \sqrt{5}}{2} * \pi * \doubleAbs{\Pi_{UDG(S)} (u,v)}
\]

\subsubsection*{Bemerkung}
UDel(S) beinhaltet nur Kanten aus dem UDG und UDel(S) ist UDG-Spanner $\rightarrow$ interessanter Subgraph\\
Aber: UDel(S) kann man nicht lokal konstruieren\\
\sieheBild{Bilder/34.png}{0.4}{bild34}{Bild 34}

\subsection*{Satz}
Die Euklidische Spanning-Ration von UGG(V) über UD(V) ist durch $\sqrt{6n-12} * \frac{1+ \sqrt{5}}{2} * \pi$ nach oben Beschränkt

\subsubsection*{Beweis}
Wir betrachten zunächst UDel(V)\\
Es sei: $xy \in UDel(V)$\\
Entweder ist $xy \in UGG$, d.h. C(x,y) ist leer\\
\sieheBild{Bilder/35.png}{0.4}{bild35}{Bild 35}
oder existiert ein eindeutiges z, sodass \[
\Delta xyz \in UDel(V)
\]
\sieheBild{Bilder/36.png}{0.4}{bild37}{Bild 36}
(Wir nennen z auch Peak-Point im Folgenden)
Aus dieser Beobachtung können wir folgenden Walk Sinnvoll definieren
\begin{equation}
SQ(x,y) = 
\begin{cases}
xy & , wenn xy \in UGG(V)\\
W(x,z) \cup SW(z,y) & , wenn xy \in UGG(V) und\, z\, ist\, Peak-Point\, zu\, xy
\end{cases} 
\end{equation}


Zunächst gilt: $|SW(x,y)| \leq 6n-12$
\\
In der Tat: Jede Kante in Udel ist adjazent zu nächsten zwei Dreiecken:\\
\sieheBild{Bilder/37.png}{0.4}{bild37}{Bild 37}
und wird damit höchstens zweimal besucht\\
\sieheBild{Bilder/38.png}{0.4}{bild38}{Bild 38}
Als planarer Graph hat UDel nach Euler-Formel maximal 3n-6 Kanten (n = Anzahl Knoten)\\
Damit ist in der Tat: \\
$|SW(x,y)| \leq 2 * (3n-6) = 6n-12$\\
Wir zeugen nun: $\doubleAbs{SW(x,y)} \leq \sqrt{|SW(x,y)|} * \doubleAbs{xy}$\\
Induktion über m = $|SW(x,y)|$\\
Für m = 1 gilt offensichtlich: \\
$\doubleAbs{SW(x,y)} = \doubleAbs{xy} = \sqrt{1} * \doubleAbs{xy}$\\
Für $m > 1:$ ist SW(x,y) = SW(x,z) $\cup$ SW(z,y) für z ist Peak-Point\\
Es sei k = SW(x,z) und damit m-k = $|SW(z,y)|$:\\
$\doubleAbs{SW(x,y)} = \doubleAbs{SW(x,z)} + \doubleAbs{z,y} \leq \sqrt{|SW(x,z)|} * \doubleAbs{xz} + \sqrt{|SW(z,y)|} * \doubleAbs{zy}$\\
$= \sqrt{k} * \doubleAbs{xz} + \sqrt{m-k} * \doubleAbs{zy}$\\
Es ist somit zu zeigen\\
$\sqrt{k} * \doubleAbs{xz} + \sqrt{m-k} * \doubleAbs{zy} \leq \sqrt{m} * \doubleAbs{xy}$ Für beliebiges $1 \leq k \leq m-1$\\
(ist beweisbar, längere Analysis Rechnung)\\
Es seien nun u und v beliebige Knoten aus V.\\
Es sei: $\Pi_{UDG}(u,v)$ euklidisch kürzester Weg in UDG\\
Es sei $\Pi_{Udel}(u,v)$ euklidisch kürzester Weg in UDel\\
Wir wissen schon $\doubleAbs{\Pi_{UDel}(u,v)} \leq \frac{1+\sqrt{5}}{2} * \pi * \doubleAbs{\Pi_{UDG}(u,v)}$\\
Außerdem existiert wie eben gezeigt für jede Kante $xy \in \Pi_{UDel}(u,v)$ existiert ein Pdad $P(x,y) \in UGG(V)$ mit $\doubleAbs{P(x,y)} \leq \sqrt{6n-12} * \doubleAbs{xy}$\\
Für den Euklidisch kürzesten Weg $\Pi_{UGG}(u,v)$ in UGG(V) gilt nun:\\
$\doubleAbs{\Pi_{UGG}(u,v)} \leq \bigcup\limits_{xy \in \Pi_{UDG}(u,v)} P(x,y)$( ein Möglicher Pfad von u nach v in UGG(V)) $= \sum\limits_{xy \in \Pi_{Udel}(u,v)} \doubleAbs{P(x,y)}$\\
$\leq \sum\limits_{xy \in \Pi_{UDel}(u,v)} \sqrt{6n-12} * \doubleAbs{xy} = \sqrt{6n-12} * \sum\limits_{xy \in \Pi_{UDel}(u,v)} \doubleAbs{xy}$ = $\sqrt{6n-12} * \doubleAbs{\Pi_{UDG}(u,v)} \leq \sqrt{6n-12} * \frac{1+\sqrt{5}}{2} * \pi * \doubleAbs{\Pi_{UDG}(u,v)}$


\section{Vorlesung 10}
\subsection*{Graph-Familie $LDel^{(K)}(V)$}
\begin{itemize}
	\item Dreieck $\triangle uvw$
	\item Dreieck $\triangle uvw$ erfüllt k-lokale Delaunay-Eigenschaft,
	wenn $C(u,v,w)$ keinen Knoten aus $N_k(u) \cup N_k(v) \cup N_k(w)$
	enthält und $\|uv\|, \|uw\|, \|vw\| \leq$ UDG-Radius
	\item $\triangle uvw$ wird dann auch als k-lokales Delaunay-Dreieck
	bezeichnet
	\item $LDel^{(k)}(V)$ enthält genau die Kanten aus dem Gabriel-Graphen
	$UGG(V)$ und alle Kanten aller möglichen k-lokalen Dreiecke \\
	{[}GG-Kanten werden zusätzlich benötigt, um Zusammenhang
	sicherzustellen. Beispiel: (Papier)]
\end{itemize}

\subsection*{Satz}
$UDel(V) \subseteq LDel^{(k)}(V)$ \\
Beweis: sei $uv \in UDel(V)$ \\
1. Fall: $\exists \triangle uvw \in UDel(V)$ (mit $uv$ ist Teil von
$\triangle uvw$) Somit ist $C(u,v,w)$ leer (d.h. enthält keine weiteren
Knoten aus $V$) und alle Kanten $uv,uw,vw$ sind in $UDG(V)$. Somit ist
für jedes $k \ \triangle uvw$ auch ein $k$-lokales Delaunay-Dreieck (für
beliebige $k$) (Papier) $N_k(u) \cup N_k(v) \cup N_k(w) \subseteq V$ \\
Folglich ist $\triangle uvw \in LDel^{(k)}(V)$ und damit auch $uv \in
LDel^{(k)}(V)$ \\
2. Fall: $uv$ ist gemeinsame Kanten von zwei Dreiecken (Papier) in
$Del(V)$ und die beiden Dreiecke $\triangle uvw$ und $\triangle uvz$
erfüllen, dass mindestens eines der Paare $(u,w)$ und $(v,w)$ bzw.
$(u,z)$ und $(v,z)$ keine UDG-Nachbarn sind. (Papier) $C(u,v,w)$ und
$C(u,v,z)$ sind leer, da $\triangle uvw$ und $\triangle uvz \in Del(V)$.
Desweiteren sind $w$ noch $z$ in $C(u,v)$ (dies würde sonst erfordern:
$\|uw\|,\|vw\| \leq \|uv\|$ und $\|uz\|,\|vz\| \leq \|uv\|$) Somit ist:
$C(u,v) \subseteq C(u,v,w) \cup C(u,v,z)$ Da $C(u,v,w)$ und $C(u,v,z)$
leer, ist auch $C(u,v)$ leer, d.h. $uv \in UGG \subseteq LDel^{(k)}(V)
\Rightarrow uv \in LDel^{(k)}(V)$ \\
3. Fall: $uv$ ist eine Kante nur eines einzigen Delaunay-Dreiecks
$\triangle uvw$ in $UDel(V)$ und mindestens eines der Paare $(u,w)$ oder
$(v,w)$ sind keine UDG-Nachbarn. \\
o.B.d.A. gelte $uw \notin UDG(V)$ (Papier) $H_1$ = Halbebene oberhalb
von L, $H_2$ = Halbebene unterhalb von L \\
Auch hier ist wie vorhin: $w \notin C(u,v)$ Damit ist $C(u,v) \cap H_1$
leer. Außerdem ist die Halbebene $L_2$ leer, da kein Delaunay-Dreieck in
dieser Ebene existiert (Delaunay-Triangulierung umfasst die konvexe
Hülle um $V$) Somit ist auch $C(u,v) \cap H_2$ leer. Damit ist $C(u,v)$
leer und somit $uv \in UGG(V) \subseteq LDel^{(k)}(V) \Box$ \\
Korollar: $LDel^{(k)}(V)$ ist ein Euklidischer $\displaystyle
\frac{1+\sqrt{5}}{2}\cdot\pi$-UDG-Spanner \\
Beweis: schon gezeigt: $UDel(V)$ ist $\displaystyle
\frac{1+\sqrt{5}}{2}\cdot\pi$-Spanner von UDG. Mit $LDel^{(k)(V)}
\supseteq UDel(V)$ folgt die Behauptung. $\Box$ \\
{[}Allgemein ist $F \subseteq G \subseteq H$ und ist $F$ ein $t$-Spanner
von $H$, dann ist auch $G$ ein $t$-Spanner.] \\
Beobachtung: $LDel^{(1)}(V)$ ist i.A. nicht planar. (Papier)
\begin{itemize}
	\item $u,v,w,x$ sehen sich gegenseitig
	\item $y$ wird von $x$ gesehen
	\item $x$ ist nicht in $C(u,v,w)$
\end{itemize}
Es gilt: $\triangle uvw$ ist ein 1-lokales Delaunay-Dreieck (kein
Nachbar von $u,v,w$ liegt in $C(u,v,w)$) Somit ist insbesondere $uv \in
LDel^{(1)}(V)$ \\
Es gilt: $xy$ ist eine GG-Kante ($C(x,y)$ ist leer) Somit ist auch $xy
\in LDel^{(1)}(V)$. $xy$ und $uv$ schneiden sich aber. 

\subsection*{PDel}
Definition: Planarized $LDel^{(1)}$ (kurz $PLDel(V)$) beinhaltet alle
Kanten aus $UGG(V)$ und nur die 1-lokalen Delaunay-Dreiecke $\triangle
uvw$, für die kein Nachbarknoten $x \in N(u) \cup N(v) \cup N(w)$
existiert, mit $xy \in LDel^{(1)}(V)$ und $y \in C(u,v,w)$ (Papier) \\

\subsection*{Lemma 1}
 Eine Gabriel-Graph-Kante $uv$ schneide ein lokales
Delaunay-Dreieck $\triangle xyz$. Dann können $u$ und $v$ nicht
gleichzeitig außerhalb von $C(x,y,z)$ liegen. „Beweis:“ (Papier)


\section{Vorlesung 11}
\subsection*{Lemma 2}
 Es seien $\triangle xyz$ und $\triangle uvw \in LDel^{(k)}(V),
k \geq 1$ \\
Die Kante $uv$ schneide $\triangle xyz$ und $C(u,v,w)$ beinhalte keinen
der Knoten $x,y,z$. (Papier) Dann beinhaltet $C(x,y,z)$ entweder $u$
oder $v$. \\
Beweis: \ldots \\

\subsection*{Satz}
$PLDel(V)$ ist planar. \\
Beweis: Zwei GG-Kanten schneiden sich nie. Ein Schnitt beinhaltet somit
immer ein 1-lokales Delaunay-Dreieck $\triangle xyz$. Annahme: Kante
$uv$ schneide $xy$. (Papier) Mit vorigen beiden Lemmata folgt: entweder
ist $u$ oder $v$ in $C(x,y,z)$. Annahme: $v$ ist in $C(x,y,z)$ \\
Mit der Dreiecksungleichung: (Papier) \\
$\|ux\| \leq \|up|+\|px\|$ \\
$\|yv\| \leq \|yp\|+\|pv\|$ \\
Somit ist: $\|ux\| + \|yv\| \leq (\|up\| + \|pv\|) + (\|xp\|+\|py\|) =
\|uv\|+\|xy\| \leq 2R$ ($R$ sei der UDG-Radius) \\
Da $\triangle xyz$ ein 1-lokales Dreieck ist, gilt: $v \notin N(x) \cup
N(y) \cup N(z)$ und somit ist $\|vy\| \geq R$ \\
Somit folgt insgesamt: $\|ux\| \leq 2R-\|vy\| < R$ \\
Somit ist $u \in N(x)$ und $uv \in LDel^{(1)}(V)$ \\
Mit $v \in C(x,y,z)$ ist damit auch $\triangle xyz$ kein Dreieck in
$PLDel(V) \lightning \Box$ \\

\subsection*{Satz}
$LDel^{(k+1)}(V) \subseteq LDel^{(k)}(V)$ \\
Beweis: Jede Kante $e \in LDel^{(k+1)}(V)$ erfüllt: \begin{enumerate}
	\item $e$ ist GG-Kante oder
	\item $\exists u,v,w$ mit $e=uv$ und $C(u,v,w)$ enthält keine Knoten aus
	$N_{k+1}(u) \cup N_{k+1}(v) \cup N_{k+1}(w)$ wegen $N_k(x) \subseteq
	N_{k+1}(x)$ folgt auch $C(u,v,w)$ enthält keinen Knoten aus $N_k(u) \cup
	N_k(v) \cup N_k(w)$.
\end{enumerate}
Also ist $e \in LDel^{(k)}(V)$ auch erfüllt. $\Box$ \\

\subsection*{Satz}
$PLDel(V) \supseteq LDel^{(2)}(V)$ \\
Beweis: Sowohl $PLDel$ als auch $LDel^{(2)}$ beinhalten alle GG-Kanten.
Damit genügt es die 2-lokalen Dreiecke aus $LDel^{(2)}$ zu betrachten.
Sei $\triangle uvw \in LDel^{(2)}(V)$. Damit gibt es keinen 1- oder
2-Hop-Nachbarn von $u,v,w$, der in $C(u,v,w)$ liegt. (Papier) \\
Somit ist insbesondere: $\cancel{\exists} x \in N_1(u) \cup N_1(v) \cup
N_1(w)$ mit $xy \in LDel^{(1)}(V)$ und $y \in C(u,v,w)$. Folglich ist
$\triangle xyz \in PLDel(V) \Box$ \\
Korollar: $LDel^{(k)}(V), k \geq 2$ ist planar. \\
Beweis: dies folgt aus $PLDel(V)$ ist planar und $PLDel(V) \supseteq
LDel^{(2)}(V) \supseteq \ldots \supseteq LDel^{(k)}(V) \Box$ 

\subsection*{RDG}
Definition: Restricted Delaunay Graph ($RDG(V)$) für $UDG(V)$ \\
$uv \in RDG(V) :\Leftrightarrow uv \in UDG(V)$ und $\forall w \in N(u)
\cap N(v): uv \in Del(N(w))$ (Papier) [Bemerkung: in dieser Definition
beinhaltet $N(u)$ auch $u$ bzw. $N(v)$ auch $v$.] \\

\subsection*{Satz}
$UDel(V) \subseteq RDG(V)$ \\
Beweis: \begin{itemize}
	\item $uv \in UDel(V) \Rightarrow uv \in UDG(V)$ und es existiert ein
	Kreis $C$ mit $u$ und $v$ sind auf dem Kreisbogen, so dass $C$ keinen
	weiteren Punkt aus $V$ enthält. (Papier)
	\item Damit enthält $\forall w \in N(u) \cap N(v)$ der Kreis $C$ auc
	keinen Knoten aus $N(w) \subseteq V$
	\item Somit ist $uv \in Del(N(w)) \forall w \in N(u) \cap N(v)$
	\item Folglich ist $uv \in RDG(V) \Box$
\end{itemize}

\subsection*{Lemma}
Sei $uv, wx \in UDG(V)$. Wenn $uv$ und $wx$ sich schneiden, dann
ist mindestens einer der Endpunkte mit allen anderen verbunden. \\
Beweis: Sei $p$ der Schnittpunkt (Papier) \\
Mit Dreiecksungleichung folgt: \\
$\|uw\| \leq \|up\| + \|pw\|$ \\
$\|vx\| \leq \|vp\| + \|px\|$ \\
Somit: $\|uw\| + \|vx\| \leq (\|up\|+\|pv\|)+(\|xp\|+\|pw\|) =
\|uv\|+\|xw\| \leq 2R$ ($R$ sei der UDG-Radius) \\
Somit ist $\|uw\| \leq R$ oder $\|vx\| \leq R$ \\
Analog ist: $\|ux\| \leq R$ oder $\|vw\| \leq R$ \\
Für alle 4 möglichen Kombinationen ist ein Punkt mit allen anderen
verbunden. $\Box$ \\

\subsection*{Satz}
$RDG(V)$ ist planar \\
Beweis: Annahme: es existieren kreuzende Kanten $uv, wx \in RDG(V)$.
(Papier) Mit $uv,wx \in RDG \Rightarrow uv, wx \in UDG$. Mit vorigem
Lemma folgt, dass mindestens einer der Knoten mit allen anderen
verbunden ist. Dies sei o.B.d.A. der Knoten $u$.
\begin{itemize}
	\item Die Kante $wx$ existiert in $RDG(V)$ nur, wenn unter Anderem für
	$u \in N(w) \cap N(x)$ gilt: $wx \in Del(N(u))$
	\item Die Kante $uv$ existiert in $RDG(V)$ nur, wenn unter Anderem für
	$u \in N(u) \cap N(v)$ gilt: $uv \in Del(N(u))$
	\item Damit ist $wx$ und $uv \in Del(N(u))$ und $wx$ kreuzt die Kante $uv$.
\end{itemize}
Das ist Widerspruch zur Planarität von $Del(N(u)) \ \Box$ \\
\subsection*{Partielle Delaunay-Triangulierung PDT(S)}
Wir betrachten UDG mit UDG-Radius $R$. Für Punktmenge $S$ verbindet die
partielle Delaunay-Triangulierung $PDT(S)$ alle Punkte $u,v \in S$ nach
folgender Regel:
\begin{enumerate}
	\item $\|uv\| \leq R$ und
	\item $uv \in GG(S) \cap UDG(S)$ oder
	\item sei $w \in S \backslash \{u,v\}$ ein Knoten, der $\alpha := \angle
	uwx$ maximiert (d.h. $\angle uwx \geq \angle uxv \forall x \in S
	\backslash\{u,v\}$) \\ es muss gelten: $C(u,v,w) \cap
	N[u]\backslash\{u,v,w\} = \emptyset$ und $\displaystyle \sin(\alpha)
	\geq \frac{\|uv\|}{R}$
\end{enumerate}
[Punkt 3 bedeutet: (Papier)]


\section{Vorlesung 12}
\subsection*{Partieller Unit-Delaunay-Graph (PuDel(S))}
\begin{itemize}
	\item Gerichtete lokale Delaunay-Kante \\
	Sei $u,v \in S$ mit $v \in N(u)$ (d.h. $\|uv\| \leq R$) \\
	$uv$ ist eine gerichtete Delaunay-Kante, wenn $uv \in Del(N(u))$
	\item Lokale Detektierbarkeit \\
	Eine gerichtete Delaunay-Kante $uv$ ist lokal detektierbar gdw.
	$\exists$ Punkt $p$, so dass folgendes gilt: $p \in VR_{N(u)}^{(u)} \cap
	VR_{N(u)}^{(v)}$ mit $\|up\| \leq \frac{R}{2}$ (Papier)
	\item Der partielle Unit-Delaunay-Graph $PuDel(S)$ beinhaltet genau alle
	lokal detektierbaren gerichteten Delaunay-Kanten
\end{itemize}
Hilfssatz 1: Für jede Punktmenge $S$ und $u,v \in S$: $uv \in PuDel(S)
\Leftrightarrow uv \in UDel(S)$ und $\exists p : p \in VR_S(u) \cap
VR_S(v)$ mit $\|up\| \leq \frac{R}{2}$
\begin{center} $\Downarrow$ \end{center}
Hilfssatz 2: Für $uv \in UDel(S)$ gilt: der direkte DT-Pfad von $u$ nach
$v$ in $Del(S)$ ist auch ein Pfad in $PuDel(S)$
\begin{center} $\Downarrow$ \end{center}
Hilfssatz 3: Für jede Kante $uv \in UDel(S)$ existiert ein Pfad
$\Pi(u,v) \in PuDel(S)$ mit $\|\Pi(u,v)\| \leq \frac{\pi}{2}\cdot\|uv\|$ \\

\subsection*{Satz}
$PuDel(S)$ ist ein Euklidischer $\frac{\pi}{2}$-Spanner von
$UDel(S)$ \\
Beweis: Betrachte $u,v \in S$. Es sei $\Pi_{UDel}(u,v) = u_0u_1\ldots
u_k$ der kürzeste Weg von $u$ nach $v$ in $UDel(S)$. Mit Hilfssatz 3
existiert für jede Kante $u_iu_{i+1} \in UDel(S)$ ein Pfad $u_i \leadsto
u_{i+1}$ in $PuDel(S)$ mit $\|u_i \leadsto u_{i+1}\| \leq \frac{\pi}{2}
\cdot \|u_iu_{i+1}\|$ (Papier) \\
Für den kürzesten Weg $\Pi_{PuDel}(u,v)$ von $u$ nach $v$ in $PuDel$
gilt: $\displaystyle \|\Pi_{PuDel}(u,v)\| \leq \|u_0 \leadsto u_1
\leadsto u_2 \leadsto u_3 \leadsto \ldots \leadsto u_k \| =
\sum_{i=0}^{k-1}\|u_i \leadsto u_{i+1}\| \leq
\sum_{i=0}^{k-1}\frac{\pi}{2}\cdot\|u_iu_{i+1}\| =
\frac{\pi}{2}\sum_{i=0}^{k-1}\|u_iu_{i+1}\| =
\frac{\pi}{2}\cdot\|\Pi_{UDel}(u,v)\|$ \\
Folglich: $\displaystyle
\frac{\|\Pi_{PuDel}(u,v)\|}{\|\Pi_{UDel}(u,v)\|} \leq \frac{\pi}{2} \
\Box$ \\
Korollar: $PuDel(S)$ ist ein Euklidischer $\displaystyle
\frac{1+\sqrt{5}}{4}\cdot\pi^2$ Spanner des UDG \\
Beweis: \\
$UDel$ ist ein $\displaystyle \frac{1+\sqrt{5}}{2}\cdot\pi$-Spanner des
$UDG$ \\
$PuDel$ ist ein $\frac{\pi}{2}$-Spanner des $UDel$ \\
Somit ist $PuDel$ ein $\displaystyle
\frac{1+\sqrt{5}}{2}\cdot\pi\cdot\frac{\pi}{2}$-Spanner des $UDG$. \\
Zeige, dass ganz allgemein gilt: Ist $F$ ein $c$-Spanner von $G$ und $G$
ein $d$-Spanner von $H$, dann ist $F$ ein $c\cdot d$-Spanner von $H$.
(Übung) 

\subsection*{CLVE}
Definition: Common Local Voronoi Edge $CLVE(u,v)$ \\
$CLVE(u,v) := VR^{(u)}_{N(u)} \cap VR_{N(u)}^{(v)} \cap VR_{N(u)}^{(u)}
\cap VR_{N(v)}^{(v)}$ (Papier) \\
Hilfssatz: $PuDel(S) \subseteq PDT(S)$ [$\|S\| \geq 3$, $S$ ist
nicht-degeneriert und nicht-colinear] \\
Beweis: zeige $uv \in PuDel(S) \Rightarrow uv \in PDT(S)$ \\
Aus $uv \in PuDel(S)$ folgt:
\begin{enumerate}
	\item $\|uv\| \leq R$ (UDG-Radius) und
	\item $\exists p \in VR_{N(u)}^{(u)} \cap VR_{N(u)}^{(v)}$ mit $\|up\| =
	\|vp\| \| \leq \frac{R}{2}$
\end{enumerate}
Betrachte $CLVE(u,v)$ \\
{[}Übung: $CLVE(u,v)$ ist keine Gerade, wegen nicht-degeneriert-Annahme.
$CLVE(u,v)$ ist kein einzelner Punkt, wegen nicht-Cozirkularität.] \\
Sonst ist $CLVE(u,v)$ entweder
\begin{enumerate}
	\item Halbgerade oder
	\item Strecke $[a,b)$
\end{enumerate}
Betrachte 2. (1. ist analog beweisbar): (Papier) \\
Es sei $m = uv \cap B(u,v)$ und $g$ die Gerade durch $u$ und $v$ \\
1. Fall: $a$ und $b$ liegen auf unterschiedlichen Seiten von $g$ und
$a,b \neq m$ (Papier) \\
Wegen $a \in VR_{N(u)}^{(u)}$ und $a \in VR_{N(v)}^{(v)}$ gilt für alle
$l \in N(u) \cup N(v)$: $\|ua\| = \|va\| \leq \|la\|$, d.h. das Innere
von Kreis $C_{\|ua\|}(a)$ ist leer. Analog is das Innere von
$C_{\|ub\|}(b)$ leer. Wegen $C(u,v) \subseteq C_{\|ub\|}(b) \cup
C_{\|ua\|}(a)$ ist damit auch $C(u,v)$ leer. Folglich ist $uv \in UDG(S)
\cap GG(S)$ und somit $uv \in PDT(S)$.

\section{Vorlesung 13}
\subsection*{Zusammenfassung der letzten Vorlesung}
PnDel = PDT\\
$uv \in Del(N(u))$\\
$\exists p \in VR_{N(u)}(u) \cap VR_{N(u)}$
$uv \in Del(S) \Rightarrow \exists u \leadsto v \in PuDel(S)$\\
$\doubleAbs{u \leadsto v} \leq \frac{\pi}{2}\doubleAbs{uv}$
\\
\sieheBild{Bilder/39.png}{0.4}{bild39}{Bild 39}
\\
PuDel = PDT\\
\grqq PaDel $\subseteq$ PDT \grqq\\
CLVE(u,v) = $VR_{N(u)}(u) \cap VR_{N(u)}(v) \cap VR_{N(v)}(u) \cap VR_{N(v)}(v)$\\
Zwei Fälle:\\
\begin{enumerate}
	\item Bild was er zu schnell weggewischt hat...
	\item a und b liegen auf einer seite von g
\end{enumerate}

\subsection*{Satz}
Es gilt: a ist Voronoi-Knoten in mindestens einer der beiden Voronoir Diagramme VD(N(u)) oder VD(N(v))\\
oBdA sei Knoten in VD(N(u)). Dann $\exists l \in N(u)$ mit a = $VR_{N(u)}(u) \cap VR_{N(u)}(v) \cap VR_{N(u)}(l)$\\
\sieheBild{Bilder/40.png}{0.4}{bild40}{Bild 40}
Der Kreis C(u,v,l) enthält keinen weiteren Knoten aus N(u). (Leere Kreisregel der dualen Delawnay-Triangulierung)
\sieheBild{Bilder/41.png}{0.4}{bild41}{Bild 41}
Da $uv \in PuDel$ gilt nach Definition:\\
$\exists p \in VR_{N(u)}(u) \cap VR_{N(u)}(v)$ mit $\doubleAbs{up} \leq R/2$ (R entspricht UDG-Radius)\\
Es ist offensichtlich: \\
$R/2 \geq \doubleAbs{up} \geq \doubleAbs{ua}$ und damit ist der Kreisdurchmesser $\leq R$\\
damit haben wir: \\
$\doubleAbs{uv} \subseteq R$\\
$sin(\alpha) \geq \frac{\doubleAbs{uv}}{R}$ \\
C(u,v,l) ist leer\\
Somit ist uv auch eine Kante in PDT(S)
\sieheBild{Bilder/42.png}{0.4}{bild42}{Bild 42}
\subsection*{Hilfssatz}
PDT(S) $\subseteq$ PuDel(S) \\

\subsubsection*{Beweis}
zeige $uv \in PDT(S) \Rightarrow uv \in PuDel(S)$\\
mit $uv \in PDT(S)$ folgt: \\
$\doubleAbs{uv} \leq R$ und \\
$uv \in GG(S) \cap UDG(S)$ oder \\
es existiert ein $l \in N(u)$ mit \\
$C(u,v,l) \cap N(u) \backslash \{u,v,l\} = \emptyset$ und \\
für $\alpha = \measuredangle u l v$ ist $sin(\alpha) \geq \frac{\doubleAbs{uv}}{R}$
Mehrere Fälle:
\begin{enumerate}
	\item $uv \in GG(S) \cap UDG(S)$\\
\sieheBild{Bilder/43.png}{0.4}{bild43}{Bild 43}
	Betrachte C(u,v)\\
	Es ist $C(u,v) \cap S \backsim \{u,v\} = \emptyset$ \\
	Somit liegt m zu keinem Punkt näher als zu u und v\\
	Wegen $N(u)  \subseteq S$ gilt diese Aussage für N(u)\\
	Somit ist $m \in VR_{N(u)}(u) \cap VR_{N(u)}(v)$\\
	Desweiteren ist $\doubleAbs{uv} \leq R$ und $\doubleAbs{um} = \doubleAbs{uv}/2$\\
	Somit ist $\doubleAbs{um} \leq R/2$\\
	Damit erfüllt uv die Def. für PuDel, d.h. $uv \in PuDel(S)$
	
	\item $\exists l \in N(u):$ \\
	$C(u,v,l) \cap N(u) \backslash \{u,v,l\} = \emptyset$ und $sin(\alpha) \geq \frac{\doubleAbs{uv}}{2}$, $\alpha = \measuredangle u l v$\\
\sieheBild{Bilder/44.png}{0.4}{bild44}{Bild 44}
	Da C(u,v,l) keine weiteren Punkte aus N(u) enthält, erfüllt der Mittelpunkt P von C(u,v,l): \\
	$\doubleAbs{up} = \doubleAbs{vp} \doubleAbs{lp} < \doubleAbs{xp}$\\
	$\forall x \in N(u) \backslash \{u,v,l\}$\\
	somit ist $p \in VR_{N(u)}(u) \cap VR_{N(u)}(v) \neq \emptyset$\\
	Mit $sin(\alpha) \geq \frac{\doubleAbs{uv}}{R} \Rightarrow$ Durchmesser von $C(u,v,l) \leq R$\\
	Somit ist $\doubleAbs{up} \leq R/2$\\
	Damit ist wieder die Vorraussetzung für PuDel erfüllt, also\\
	$uv \in PuDel(S)$\\
	Korollar: PuDel(S) = PDT(S)\\
	Beweis folgt unmittelbar aus den beiden Hilfssätzen. 
	
\end{enumerate}

\subsection*{Zusammenfassung}
Im folgenden: \circlearound{xxx} um ein Vorkommen des Graph entspricht nicht lokal konstruierbar
\vspace*{10mm}
\\
\noindent
\circlearound{EMST} $\subseteq$ LMST (maximaler Grad ist 6) $\subseteq$ URNG (keine Grad-Beschränkung) $\subseteq$ RNG $\subseteq$ UGG $\subseteq$ GG
\vspace*{10mm}
\\
URNG $\leftarrow$ Euklidischer $\Theta(n)$ Spanner des UDG\\
UGG $\leftarrow$ Euklidischer $\Theta(\sqrt{n})$ Spanner des UDG
\vspace*{10mm}
\\
UGG $\subseteq$ PDT $\subseteq$ UDel
\vspace*{10mm}
\\
PDT = PuDel $\leftarrow$ Eukldischer $\frac{1+ \sqrt{5}}{4} * \pi^2$ UDG-Spanner
\vspace*{10mm}\\

\circlearound{UDel} $\subseteq$ $LDel^{(k)} \subseteq LDel^{(k-1)} \subseteq ... \subseteq LDel^{(2)}$ 
\vspace*{10mm}
\\
UDel $\leftarrow$ Euklidischer $\frac{1 + \sqrt{5}}{2} * \pi$ UDG-Spanner\\
PLDel $\subseteq LDel^{(1)} \subseteq UDG$
(Von Anfang bis PLDel ist alles planar)
\vspace*{10mm}
\\
$UDel \subseteq RDG \subseteq UDG$\\
$UDel \subseteq \circlearound{Del}$\\
Del $\leftarrow$ Euklidischer $\frac{2\pi}{3 cos(\pi / 6)}$ Spanner des vollständigen euklidischen Graphen
UDel $\leftarrow$ Euklidischer $\frac{1 + \sqrt{5}}{2}$ UDG-Spanner

\subsubsection*{Original}
\sieheBild{Bilder/45.png}{0.4}{bild45}{Bild 45}

\section{Vorlesung n-2}
\subsection*{Motivation}
\begin{enumerate}
	\item Nur anhand des Pfades werden Nachrichten erzeugt. \sieheBild{Bilder/46.png}{0.4}{bild46}{Bild 46}
	\item a) Lokaler Broadcast \\
			b) alle nachbarn antworten\\
			c) Berechne PDT lokal\\
			\sieheBild{Bilder/47.png}{0.4}{bild47}{Bild 47}
	\item Besser wäre \\
			a) lokaler Broadcast\\
			b) idealerweise antworten nur die PDT-Nachbarn $\rightarrow$ PDT Graph damit schon konstruiert\\
			das ist dann im Idealfalle Nachrichtenoptimal
			HIEER KÖNNTE IHR BILD STEHEN
\end{enumerate}

\subsection{Zur Definition des reaktiven PDT-Algorithmus}
eine (beweisbar) äquivalente PDT-Definition: \\
$\forall uv \in UDG$ gilt : $uv \in PDT \Leftrightarrow \forall w \in C(u,v)$ gilt:
\begin{enumerate}
	\item $C(u,v) \cap \overline{H^w(u,v)}$ ist leer und
	\item $sin(\alpha) \geq \doubleAbs{uv} \text{ mit } \sin \alpha  \measuredangle uwv$
\end{enumerate}
[ Hierbei ist $H^(u,v)$ die Halbebene bzgö u und v die w enthält]\\
\sieheBild{Bilder/48.png}{0.4}{bild48}{Bild 48}
\\
$\overline{H^w(u,v)}$ ist das Komplement von $H^(u,v)$
\\
Algorithmusbeschreibung: Reactive PDT
\begin{enumerate}
	\item Start\\
	\sieheBild{Bilder/49.png}{0.4}{bild49}{Bild 49}
	\\
	\begin{itemize}
		\item sende RTS blind per Broadcast an alle Nachbarn
		\item setze Time t(n) auf $t_{max}$ [$t_{max}$ ist ein Parameter; beliebige aber feste Wartezeit bis das Verfahren fertig ist]
	\end{itemize}
	\item RTS-Empfang in einem Knoten v:\\
	u $\overset{RTS}{\rightarrow}$ v
	\\
	\begin{itemize}
		\item Initialisiere Menge der bekannten Knoten $S(v) = \emptyset$
		\item setze Maximalwinkel auf $\pi / 2$\\
		$\alpha_{max} (v) = \pi/ 2$
		\item setze Timer t(v) auf $\doubleAbs{uv} * t_{max}$ [Annahme: UDG Radius ist 1, ansonsten setze auf Abstand $\frac{\doubleAbs{uv}}{R} * t_{max}$]
	\end{itemize}
	\item Timer t(v) in Knoten v abgelaufen:\\
	\sieheBild{Bilder/50.png}{0.4}{bild50}{Bild 50}
	\begin{itemize}
		\item sende CTS mit Positioninfo von v per Broadcast an alle Nachbarn von v (damit wird natürlich auch unter anderem der Ursprungsknoten u erreicht)
	\end{itemize}
	\item Knoten v hört CTS von Knoten z\\
	\sieheBild{Bilder/51.png}{0.4}{bild51}{Bild 51}
	\begin{itemize}
		\item Ergänze z in der Menge der bekannten Nachbarn S(v) = $S(v) \cup \{z\}$
		\item überprüfe ob uv bzgl. S(v) die PDT Bedingung verletzt. \begin{itemize}
			\item falls ja: Lösche Timer t(v) (d.h. v wird kein cts generieren)
			\item falls nein: Testen ob $\alpha = \measuredangle uzv > \alpha_{max}(v)$
			\begin{itemize}
				\item falls ja: $\alpha_{max} = \alpha$, aktualisiere Timer t(v) auf $\frac{\doubleAbs{uv}}{sin(\alpha)} * t_{max}$ [d.h. der Timeout wird proportional zum Durchmesser von Kreis um    C(u,z,v) gesetzt][Annahme ist UDG Radius ist 1; ansonsten setze Timer auf $\frac{\doubleAbs{uv}}{sin(\alpha)} * \frac{1}{R} * t_{max}$\\
				Es ist hier  $\frac{\doubleAbs{uv}}{sin(\alpha)} \leq R$]
			\end{itemize}
		\end{itemize}
	\end{itemize}
	\item u empfängt CTS von v\\
	$u \overset{CTS}{\leftarrow} v$
	\begin{itemize}
		\item Füge v zur Liste der PDT Nahbarn hinzu
		\item Wenn Timeout t(u) abgelaufen ist, ist die Menge der PDT Nachbarn vollständig
	\end{itemize}
\end{enumerate}
\subsubsection{Korrektkeit}
v sendet CTS $\Leftrightarrow$ uv ist eine PDT-Kante

\section{Vorlesung n-1}
\sieheBild{Bilder/52.png}{0.4}{bild52}{Bild 52}
$\forall uv \in UDG$ \\
$uv \in PDT \Leftrightarrow \forall uv \in C(u,v)$\\
\sieheBild{Bilder/53.png}{0.4}{bild53}{Bild 53}

\subsection{Alternative Algorithmenform zur reaktiven KOnstruktion von lokaler Sicht auf planaren Graphen}

A) Selection-Phase
\begin{enumerate}
	\item Start \\
	\sieheBild{Bilder/54.png}{0.4}{bild54}{Bild 54}
	\begin{itemize}
		\item Sende RTS per Braodcast mit eigeneer Position an alle Nachbarn
		\item setze Timer t(u) = $t_max$
	\end{itemize}
	\item bei RTS-Empfang\\
	$u \overset{RTS}{\rightarrow} v$\\
	\begin{itemize}
		\item setze timer t(v) = $\frac{\doubleAbs{uv}}{R} * t_max$ [bzw.  vereinfacht t(v) = $\doubleAbs{uv}$ wenn UDG-Radius R = 1]
	\end{itemize}
		
	\item bei Ablauf von Timer t(v) sendet v ein CTS
	\item Wenn v ein CTS von w empfängt und $w \in C(u,v)$ d.h. v ist kein GG-Nachbar von u), dann lösche Timer t(v) \\
	\sieheBild{Bilder/55.png}{0.4}{bild55}{Bild 55}
	
	ISt der Algorithmus an dieser Stelle fertig?
	\\
	\sieheBild{Bilder/56.png}{0.4}{bild56}{Bild 56}
	\begin{itemize}
		\item Timeouts werden in der Reihenfolge $w_1,w_2,w_3,u$ gesetzt.
		\item $w_1$ unterdrückt $w_2$
		\item damit sendet $w_3$ ein CTS
		\item Somit enthält das Ergebnis zwar alle GG-Kanten aber möglicherweise noch weitere Kanten (hier: $uw_1 \in GG$ aber auch $uw_3 \not\in GG$)
	\end{itemize}
	\item Wenn v ein CTS von w empfängt und Timer von v schon gelöscht, dann: \\
	Falls $w \in C(u,v)$ \\
	S(v) = S(v) $\cup \{w\}$\\
	\sieheBild{Bilder/57.png}{0.4}{bild57}{Bild 57}
\end{enumerate}
B) Protest-Phase: nach Ablauf der Selection-Phase (d.h. nach $t_max$) statt
\begin{enumerate}
	\item Start: Setze Timer t(v) = $\frac{\doubleAbs{uv}}{R} * t_max$ (d.h. die am nächsten liegenden Knoten protestieren zuerst)
	\item Bei Ablauf von Timer t(v) falls S(v) $\neq \emptyset$\\
	Sende Protestnachricht an Knoten u
	\item Startknoten u empfängt Protest-Nachricht von v:
	\\
	u entfernt alle bisherigen Kanten uw mit $v \in C(u,w)$\\
	\sieheBild{Bilder/58.png}{0.4}{bild58}{Bild 58}
	\item (zur möglichen Reduktion der Anzahl Protestnachrichten) Knoten v empfängt Protest von w.
	\\
	v entfernt alle x aus S(v) mit $w \in C(u,x)$\\
	\sieheBild{Bilder/59.png}{0.4}{bild59}{Bild 59}
\end{enumerate}

\subsection{formale Charakterisierung der behandelten reaktiven Verfahren im Sinne von Nachrichtenkomplexität}
Für RPDT gilt: \\
\sieheBild{Bilder/60.png}{0.4}{bild60}{Bild 60}
1 Nachricht für das RTS (beachte: Broadcast Medium)\\
\sieheBild{Bilder/61.png}{0.4}{bild61}{Bild 61}
genau eine Nachricht für jeden PDT Nachbarn\\
Nachrichtengröße: 
\begin{itemize}
	\item eigene Adresse: O(log n) Bits bei n Knoten
	\item eigene Position: O(1) Bits
\end{itemize}
Betrachte PDT-Konstruktion als Topologieabbildung\\
$\tau : UDG \rightarrow PDT$
\\
Sei $G \in UDG$. Zur konstruktion der lokalen Sicht. $\tau (G)[v]$ von v auf $\tau (G)$ werden $ (1 +  | \tau (G)[v] |) * O(log(n) + 1) = O(|\tau(G)[v]| * log(n)$ \\
Dabei ist $| \tau (G)[v] |$  die Anzahl der Nachbarn von v in $\tau (G)$\\
$ 1 +  | \tau (G)[v] | $ Anzahl der Nachrichten\\
$O(log(n) + 1) $ Grße der Nachricht\\

\subsubsection{Verallgemeinerung auf}
\begin{itemize}
	\item Verfahren die auch virtuelle Topologien erzeugen
	\\
	\sieheBild{Bilder/62.png}{0.4}{bild62}{Bild 62}
	\\
	virtueller Knoten v im virtuellen Graph\\
	u ist Proxy im realen Netzgraph, der die Sicht auf v konstruiert\\
	\item Verfahren, die mehr als die Ein-Hop-Nachbarschaft der konstruierten Topologie benötigen; d.h. K-Hop-Nachbarschaft für ein festes k.
\end{itemize}

\subsection{Definition}
lokale Topologiekontrolle: $\tau: \mathcal{C} \rightarrow \mathcal{D}$ ist $O_k$ reaktiv wenn stets gilt: Sei $G \in \mathcal{C}$ mit n Knoten $\forall v \in \tau(G) \exists u \in G$, sodass $\tau(G)[v]$ wie folgt konstruierbar ist
\begin{itemize}
	\item Kein Knoten aus G kennt zu Beginn seine Nachbarn
	\item Knoten u sendet zu Beginn einen Broadcast
	\item die Nachrichtenkomplexität des Verfahrens ist $O(|\tau(G) [v,k]| * log(n))$, mit $|\tau(G)[v,k]|$ ist die Anzahl der Knoten in $\tau(G)[v,k]$
\end{itemize}

\subsubsection{Ist BFP-GG (Verfahren von eben) O-reaktiv?}
\sieheBild{Bilder/63.png}{0.4}{bild63}{Bild 63}
Es ist: $\doubleAbs{vw_1} < \doubleAbs{vw_2} < \doubleAbs{vw_3}$ und $C(v,w_1) = \emptyset, w_1 \in C(w,w_2), w_2 \in C(v,w_3)$\\
Betrachte $vw_3$ wie eben die Kante $vw_1$ und führe dieselbe Konstruktion durch.\\
Setze beliebig fort: Man erhält: \\
\sieheBild{Bilder/64.png}{0.4}{bild64}{Bild 64}
mit $\doubleAbs{vw_i} < \doubleAbs{vw_{i+1}} und w_i \in C(v, w_{i+1})$
Damit ist $GG(\{v, w_1, ..., w_k\})[v] = \{vw_1\}$\\
Aber Anzahl Nachrichten: 
\begin{itemize}
	\item Broadcast
	\item Nachricht $w_1 \rightarrow w$ schaltet $w_2$ ab. Es folgt:
	\item Nachricht $w_3 \rightarrow v$ schaltet $w_4$ ab. 
	\item usw. 
\end{itemize}
Also ist die ANzahl der Nachrichten O(K) und nicht O(1), was hier wegen der einzigen GG-Kante aber für O-Reaktivität notwendig wäre. \\
Dennoch gilt: BFP-GG ist bzgl. Nachritenkomplexität besser als erst alle Nachbarn zu ermitteln.\\
Beispiel: \\
\sieheBild{Bilder/65.png}{0.4}{bild65}{Bild 65}
\\
Plaziere für beliebiges K Knoten in diese Halbebene.
\\
Es ist $v \in C(u,v_i)$ und $\doubleAbs{v_i u} > \doubleAbs{uv}$
Somit antwortet v auf RTS von u zuerst. Alle anderen $v_i$ schalten Timer ab. Keine Protestnachrichten werden verschickt.
Die Anzahl der Nachrichten ist damit 2 anstatt  die O(k) vielen Nachrichten, die erst mal erforderlich sind, um alle K-Hop-Nachbarn zu bestimmen.


\end{document}


